\documentclass[a4paper,dvipdfmx,11pt]{article}
\usepackage[T1]{fontenc}
\usepackage{geometry}
\usepackage[straightquotes,ttzdefault]{newtxtt}
\usepackage{framed}
\usepackage[dvipsnames]{xcolor}
\usepackage{colorframed}
% we put this here but it does not have to be in preamble of
% course
% titled-frame
\colorlet{TFFrameColor}{teal!20}
\colorlet{TFTitleColor}{purple}
\def\colorframedTFconlabcolorcommand{\color{purple}}
% shaded
\colorlet{shadecolor}{orange!10}
% framed
\FrameRule5pt\relax
\FrameSep9pt\relax % this is framed.sty default anyhow, 3\fboxsep
% attention that \FrameSep influences framed but also shaded, shaded* and
% oframed, which is not so clear from reading only framed documentation
\def\colorframedbordercolorcommand{\color{red!20}}

\title{The \colorframed package%
\vadjust{\vtop to 0pt{\hbox
    to\linewidth{\strut\footnotesize Release 0.9a of
      2022/09/23\hss report issues at \url{https://github.com/jfbu/colorframed}}\vss}}%
}
\author{Jean-François B.}
\date{}
\usepackage{shortvrb}\MakeShortVerb\|
%\usepackage{url}\urlstyle{same}
\usepackage{hyperref}
\def\ctanpackage#1{\href{https://ctan.org/pkg/#1}{#1}}

\usepackage{amssymb}

\usepackage{parskip}
\usepackage{xspace}

\definecolor{joli}{RGB}{225,95,0}
\definecolor{JOLI}{RGB}{225,95,0}
\newcommand\colorframed{%
        \texorpdfstring{{\color{joli}\bfseries colorframed}}{colorframed}\xspace}


{\makeatletter\sbox0{\color{RawSienna}\xdef\foo{\current@color}}}
     \edef\verbatimpushcolor{\special{color push \foo}}
     \def\verbatimpopcolor{\special{color pop}}
\AddToHook{env/verbatim/before}{\verbatimpushcolor}
\AddToHook{env/verbatim/end}{\verbatimpopcolor}
\begin{document}
\maketitle

\section{Description}

This package fixes various colour leaks one encounters with
the environments from Donald Arseneau's package
\ctanpackage{framed}.  Typically, colour leaks occur if using
|\color| (at top level) inside the environments, or more subtly also when
using |\textcolor| with an argument ending up being split at a
page break.

This latter type of colour leak (or colour disappearance) is the
more challenging one as the fix requires modifications or
replacements not only of some of the \ctanpackage{framed}.sty
macros (such as its |\CustomFBox|, which \colorframed overwrites)
but also to some \LaTeX2e internals, as some environments of
\ctanpackage{framed}.sty rely on usage of |\fbox| or |\colorbox|.
Rather than overwriting internal \LaTeX2e macros such as
|\@frameb@x| or |\color@b@x|, \colorframed simply replaces |\fbox|
and |\colorbox| in the \ctanpackage{framed}.sty environments by
appropriate substitutes.

I am aware \ctanpackage{tcolorbox} package documentation explains at
least one colour issue which looks similar to those fixed
here in \ctanpackage{framed} context, and that the fix overthere uses
an extra colour stack, hence is not xelatex compatible
currently.

The problems are solved here without involving an extra
colour stack, hence the fixes work also with xelatex.

% The analysis and original workarounds for using framed.sty
% with colours were developed by me in some contributions I made
% to the Sphinx project (\url{https://github.com/sphinx-doc/sphinx})
% and I am transferring here the general idea.

% The key thing is that the boxes handled by framed.sty may
% contain isolated colour push or colour pop.  We must make
% sure an isolated colour push, if followed by colour changes,
% is always followed by paired ones, and never by a colour pop
% from a colour command originated "prior".

% TODO: transfer also the breakable box environment based on pict2e
% and allowing rounded corners and shadows developped overthere.

\section{The environments from \ctanpackage{framed}}

\begin{titled-frame}{A list of the environments from package \ctanpackage{framed}}
  This list indicates which boxing macros are
  used in the original, and their replacement.
% (investigate later why something such as shortverb |#| in
% preamble does not work)
\halign{\bfseries#\hfil&(#\unskip)~\hfil&#\hfil\cr
%
  framed & |\fbox| & Replaced by specialization of own \rlap{\string\CustomFBox.}\cr
%
  oframed & |\CustomFBox| & \colorframed uses a redefined |\CustomFBox|.\cr
%
  titled-frame & |\CustomFBox| & \emph{id.}\cr
%
  shaded & |\colorbox| & Replaced by |\colorframedcolorbox|.\cr
%
  shaded* & |\colorbox| & \emph{id.}\cr
%
  snugshade & |\colorbox| & \emph{id.}\cr
%
  snugshade* & |\colorbox| & \emph{id.}\cr
%
  leftbar & none & This one does not use any boxing macro.\cr
}
\end{titled-frame}

\begin{shaded*}
  We refer the reader to \ctanpackage{framed} documentation
  and provide here only a few additional details, particularly
  regarding the `titled-frame' environment as it is described
  in \ctanpackage{framed} documentation more as being a
  template than a user-level finalized environment.  The above
  box gives an example of its use.  It is an environment with
  one mandatory argument which provides the title of the
  frame, which is repeated after a page break with |(cont)|
  appended.  The colours |TFFrameColor| and
  |TFTitleColor| must be defined by user.  To customize
  further one will need to renew the environment definition,
  which is left untouched by \colorframed (which modifies
  rather |\TitleBarFrame| and |\CustomFBox|).  Here is
  how this environment is defined inside \ctanpackage{framed}.sty: (code and comments by Donald Arseneau)\par
\begin{footnotesize}
\begin{verbatim}
% A particular type of titled frame with continuation marks.  
% Parameter #1 is the title, repeated on each page.
\newenvironment{titled-frame}[1]{%
  \def\FrameCommand{\fboxsep8pt\fboxrule2pt
     \TitleBarFrame{\textbf{#1}}}%
  \def\FirstFrameCommand{\fboxsep8pt\fboxrule2pt
     \TitleBarFrame[$\blacktriangleright$]{\textbf{#1}}}%
  \def\MidFrameCommand{\fboxsep8pt\fboxrule2pt
     \TitleBarFrame[$\blacktriangleright$]{\textbf{#1\ (cont)}}}%
  \def\LastFrameCommand{\fboxsep8pt\fboxrule2pt
     \TitleBarFrame{\textbf{#1\ (cont)}}}%
  \MakeFramed{\advance\hsize-20pt \FrameRestore}}%
%  note: 8 + 2 + 8 + 2 = 20.  Don't use \width because the frame title
%  could interfere with the width measurement.
 {\endMakeFramed}
\end{verbatim}
% pas de \footnote dans les environnements de framed...
\normalcolor
Side note: this `titled-frame' environment was in effect
broken in recent \LaTeX\ which has modified how |\smash|
behaves; \colorframed fixes this infelicity in passing.\par

\emph{(the current box is an example of \emph{`shaded*'} environment, next one is with
\emph{`snugshade*'}, both use a \emph{shadecolor} which must be defined by user)}
\par
\end{footnotesize}
\end{shaded*}
% jeudi 22 septembre 2022 à 14:37:08
% Il se passe des choses très bizarres avec snugshade* ou
% framed en premier après un \item qui me semblent être des
% bugs de framed.  Dans Sphinx, je n'ai pas eu ces problèmes,
% au contraire je suis sûr que ça fonctionne avec les
% environnements que j'y ai définis reposant sur framed (et un
% \trivlist en général), j'ai testé plusieurs fois.
% Ce n'est pas induit par colorframed j'ai vérifié sans lui.
%\begin{enumerate}
%\item\relax% {\footnotesize Now testing the `snugshade*' environment.}
\begin{snugshade*}
\leavevmode\color{ForestGreen}
  One does not need to dive into the details of the macros
  used above to understand intuitively how they are supposed
  to influence the final output.  To modify this output,
  simply redefine this environment with suitable changes.

  Notice in particular that |\blacktriangleright|
  (which produces $\blacktriangleright$ and acquires thus a colour
  despite its name) requires to
  the best of my knowledge loading \ctanpackage{amssymb} or
  some other math symbols package and it is up to user to do it, if
  its usage is kept.
%
  The original environment gives to this continuation label the
  same colour as the frame.  The \colorframed variant adds the
  possibility to customize this colour via suitably defining a
  macro, in the example above we did:
\begin{verbatim}
\renewcommand\colorframedTFconlabcolorcommand{\color{purple}}
\end{verbatim}
\end{snugshade*}

\begin{framed}
  This is an example of usage of the environment `framed'.

  This environment allows customization of the border width
  and of the separation with contents, via |\FrameRule| and
  |\FrameSep|, but not of the colour of the border.  The
  \colorframed version adds this possibility: it is simply a
  matter of redefining the |\colorframedbordercolorcommand|
  macro, which defaults to |\normalcolor|.

\footnotesize (|\color{blue}| in source)
\normalsize
\color{blue}
  So the current frame was configured using:
\begin{verbatim}
\setlength{\FrameRule}{5pt}
\setlength{\FrameSep}{9pt}
\renewcommand\colorframedbordercolorcommand{\color{red!20}}
\end{verbatim}
\begin{footnotesize}\normalcolor
  The length |\FrameSep| influences also `oframed', `shaded', and `shaded*',
  but `snugshade' and `snugshade*' employ |\fboxsep| rather.

  As per |\FrameRule|, it influences `framed' and `oframed' but not
  `titled-frame' which simply uses |\fboxrule| for its border width.\par
\end{footnotesize}
  The text colour induced from a |\color{blue}|
  will not leak out to the frame or to the text following this
  environment, even in case of a pagebreak.
\end{framed}

As indicated the aim of \colorframed is not to modify the
existing environments of \ctanpackage{framed}.sty into
acquiring more capabilities and customizability, and we have
indicated already the two sole customization additions. The aim
was strictly to fix the colour leak issues.

\section{TODO}

The author has developed based upon usage of
\ctanpackage{pict2e} breakable boxes with round corners,
background colour, optional shadow, and other goodies and is
planning on incorporating this environment into the package.
%
Of course, it will remain limited in comparison to the fully
customizable boxes provided by package \ctanpackage{tcolorbox}
but our testing showed significant speed-up in build time, which
may matter for long documents.

Hopefully this addition will be done when time will permit.

\thispagestyle{empty}
\enlargethispage{3\baselineskip}

\footnotesize After initial release made it to CTAN on 2022/09/22
I became aware of \ctanpackage{longfbox} which provides already
such \ctanpackage{pict2e} breakable boxes with rounded corners
(even elliptical arcs), and furthermore with a CSS-like interface
which is exactly what I had done on my side too... I need to check
more \ctanpackage{longfbox} before a decision is made here!
Perhaps it is best to actually keep \colorframed as it is and do
another package for extras.

\centerline{\hrulefill documentation ends here\hrulefill}

\end{document}