\documentclass[a4paper,dvipdfmx,10pt,english]{article}
\usepackage{babel}
\usepackage[T1]{fontenc}
\usepackage{geometry}
\usepackage[straightquotes,ttzdefault]{newtxtt}
\usepackage{framed}
\usepackage[dvipsnames]{xcolor}
\usepackage{colorframed}
% we put this here but it does not have to be in preamble of
% course
% titled-frame
\colorlet{TFFrameColor}{teal!20}
\colorlet{TFTitleColor}{purple}
\def\colorframedTFconlabcolorcommand{\color{purple}}
% shaded
%\colorlet{shadecolor}{orange!10}
\colorlet{shadecolor}{lightgray!25}

% framed
\FrameRule5pt\relax
\FrameSep9pt\relax % this is framed.sty default anyhow, 3\fboxsep
% attention that \FrameSep influences framed but also shaded, shaded* and
% oframed, which is not so clear from reading only framed documentation
\def\colorframedbordercolorcommand{\color{red!20}}

\title{The \colorframed package%
\vadjust{\vtop to 0pt{\hbox
    to\linewidth{\strut\footnotesize Release 0.9b of
      2022/10/01\hss report issues at \url{https://github.com/jfbu/colorframed/issues}}\vss}}%
}
\author{Jean-François B.}
\date{}
\usepackage{shortvrb}\MakeShortVerb\|
%\usepackage{url}\urlstyle{same}
\usepackage{hyperref}
\def\ctanpackage#1{\href{https://ctan.org/pkg/#1}{#1}}

\usepackage{amssymb}

\usepackage{parskip}
\usepackage{xspace}
\usepackage{metalogo}
\setlogokern{La}{-0.258em}
\setlogokern{aT}{-0.08em}
\setlogokern{eL}{-0.1em}
%\setlogokern{X2}{0em}
\usepackage{enumitem}

\definecolor{joli}{RGB}{225,95,0}
\definecolor{JOLI}{RGB}{225,95,0}
\newcommand\colorframed{%
        \texorpdfstring{{\color{joli}\bfseries colorframed}}{colorframed}\xspace}


% ATTENTION: dvipdfmx syntax here!
{\makeatletter
     \sbox0{\color{RawSienna}%
            \xdef\verbatimpushcolor{\special{color push \current@color}}}%
}
\def\verbatimpopcolor{\special{color pop}}
\AddToHook{env/verbatim/before}{\verbatimpushcolor}
\AddToHook{env/verbatim/end}{\verbatimpopcolor}

\newcommand\ghissue[2][]{\href{https://github.com/jfbu/colorframed/issues/#2}{#1\##2}}

\makeatletter
  %\show\verbatim@font % pas changé at begin doc
  \def\verbatim@font{\normalfont\ttfamily\color{RawSienna}}%
\makeatother

\begin{document}
\maketitle

\vskip-1.75\baselineskip

{\footnotesize
\halign{\tabskip1em\bfseries#\unskip&\hfil(#\unskip)\hfil&#\unskip\hfil\cr
0.9  & 2022/09/22 & initial release\cr
0.9a & 2022/09/23 & doc fixes, mention \ctanpackage{longfbox}\cr
0.9b & 2022/10/01 & - fix "placement of titled-frame title
                                  is not backwards compatible" (\ghissue{2})\cr
\omit&\omit& - |\colorframedTitleBarFrame| as work around to native issues with titled-frame
               (\ghissue{3}, \ghissue{4})\cr
\omit&\omit& - link user to \ghissue{5} for usage with list items, other doc improvements
                (such as \ghissue{1})\cr
}}

\vskip-\baselineskip\vskip0pt\relax

\section{Description}


This package fixes various colour leaks one encounters with
the environments from Donald Arseneau's package
\ctanpackage{framed}.  Typically, colour leaks occur if using
|\color| (at top level) inside the environments, or more subtly also when
using |\textcolor| with an argument ending up being split at a
page break.

This latter type of colour leak (or colour disappearance) is the
more challenging one as the fix requires modifications or
replacements not only of some of the \ctanpackage{framed}.sty
macros (such as its |\CustomFBox|, which \colorframed overwrites)
but also to some \LaTeX{} internals, as some environments of
\ctanpackage{framed}.sty rely on usage of |\fbox| or |\colorbox|.
Rather than overwriting internal \LaTeX{} macros such as
|\@frameb@x| or |\color@b@x|, \colorframed simply replaces |\fbox|
and |\colorbox| in the \ctanpackage{framed}.sty environments by
appropriate substitutes.

I am aware \ctanpackage{tcolorbox} package documentation explains at
least one colour issue which looks similar to those fixed
here in \ctanpackage{framed} context, and that the fix overthere uses
an extra colour stack, hence is not \XeLaTeX{} compatible
currently.

The problems are solved here without involving an extra
colour stack, hence the fixes work also with \XeLaTeX.

\section{The environments from \ctanpackage{framed}}

  We refer the reader to \ctanpackage{framed} for the official
  documentation and provide here only a few relevant details,
  particularly
  regarding the `titled-frame' environment which is described
  in \ctanpackage{framed} documentation more as being a
  template than a user-level finalized environment.
\begin{titled-frame}{A list of the environments from package \ctanpackage{framed}}
  This list indicates which boxing macros are
  used in the original, and their replacement.
% (investigate later why something such as shortverb |#| in
% preamble does not work)
\halign{\bfseries#\hfil&(#\unskip)~\hfil&#\hfil\cr
%
  framed & |\fbox| & Replaced by specialization of own \rlap{\expandafter|\string\CustomFBox|.}\cr
%
  oframed & |\CustomFBox| & \colorframed uses a redefined |\CustomFBox|.\cr
%
  titled-frame & |\CustomFBox| & \emph{id.}\cr
%
  shaded & |\colorbox| & Replaced by |\colorframedcolorbox|.\cr
%
  shaded* & |\colorbox| & \emph{id.}\cr
%
  snugshade & |\colorbox| & \emph{id.}\cr
%
  snugshade* & |\colorbox| & \emph{id.}\cr
%
  leftbar & none & This one does not use any boxing macro.\cr
}
\end{titled-frame}
The previous box gives an example of use of `titled-frame' environment.
  It is an environment with
  one mandatory argument which provides the title of the
  frame, which is repeated after a page break with |(cont)|
  appended.  The colours |TFFrameColor| and
  |TFTitleColor| must have been defined by the user.
  \colorframed adds the possibility to colorize separately the continuation
  label (else it inherits the colour of the frame).  For example we used
\begin{verbatim}
\renewcommand\colorframedTFconlabcolorcommand{\color{purple}}
\end{verbatim}
  and the default definition of this macro is to do nothing.

\medskip
% check why topsep influences separation of item 1 and 2
% it shouldn't? hmm... ah yes I think framed itself uses \topsep...
%\begin{enumerate}[noitemsep,topsep=6pt,font=\normalcolor]
% j'utilise enumitem+noitemsep sinon je vois dans la partie ombrée
% de l'espace supplémentaire au début de l'item (peut-être aussi
% lié à l'emploi du package parskip)
% \colorlet{shadecolor}{lightgray!25}

\begin{footnotesize}
\begin{itemize}[noitemsep,topsep=3pt,% influences strangely item separation
                         parsep=3pt]
\begin{shaded}
\item
\emph{The current box is an example of \emph{`shaded'}
    environment; The
    |shadecolor| must have been defined by user. See the
    \ghissue[repository issue ]{5} for some comments about how to use
    such environment with list items, if the environment is to start
    at start of the item. }
\end{shaded}

\begin{snugshade}
\item \emph{This is an example of \emph{`snugshade'}
    environment. It shares with \emph{`shaded'} the usage of |shadecolor|.}
\end{snugshade}

\begin{snugshade*}
\item \emph{This is an example of \emph{`snugshade*'}.}
\end{snugshade*}

\begin{snugshade}
\item
%
  The `titled-frame' environment was in effect broken in recent \LaTeX\ which
  has modified how |\smash| behaves (the continuation label created a blank
  line interrupting the framing); \colorframed fixes this infelicity in
  passing.

  To customize
  further usage of `titled-frame' one must renew its definition,
  which is left untouched by \colorframed.
  Here is its source from \ctanpackage{framed}.sty: (code and
  comments by Donald Arseneau)%
\begin{verbatim}
% A particular type of titled frame with continuation marks.  
% Parameter #1 is the title, repeated on each page.
\newenvironment{titled-frame}[1]{%
  \def\FrameCommand{\fboxsep8pt\fboxrule2pt
     \TitleBarFrame{\textbf{#1}}}%
  \def\FirstFrameCommand{\fboxsep8pt\fboxrule2pt
     \TitleBarFrame[$\blacktriangleright$]{\textbf{#1}}}%
  \def\MidFrameCommand{\fboxsep8pt\fboxrule2pt
     \TitleBarFrame[$\blacktriangleright$]{\textbf{#1\ (cont)}}}%
  \def\LastFrameCommand{\fboxsep8pt\fboxrule2pt
     \TitleBarFrame{\textbf{#1\ (cont)}}}%
  \MakeFramed{\advance\hsize-20pt \FrameRestore}}%
%  note: 8 + 2 + 8 + 2 = 20.  Don't use \width because the frame title
%  could interfere with the width measurement.
 {\endMakeFramed}
\end{verbatim}
  The continuation label used in the `titled-frame' environment
  is |$\blacktriangleright$|
  (which despite its name reacts to current text colour setting).
  It requires to
  the best of the author knowledge loading \ctanpackage{amssymb} or
  some other math symbols package and it is up to user to do it.
\end{snugshade}

% \begin{shaded} % nesting does not work
\begin{framed}
\item
\emph{(one can not nest environments of \ctanpackage{framed}.sty, else we would
  have done so here.)}

\textcolor{blue}{(Here is an attempt to use the {`framed'} environment
  around a list item; we can see an unexpected
  induced indentation of the
  paragraph. And, by the way, there is no {`snugframe'} environment which would
  let the border fit not the whole width but only the indented item.)}
\end{framed}
% \end{shaded} % nesting does not work

\begin{snugshade}
\item To customize the `titled-frame' output, one may simply redefine the
  environment via suitable changes e.g. to the setting of |\fboxsep| and
  |\fboxrule|.  But doing so, one quickly discovers that the layout from
  |\TitleBarFrame| has problems (see on this the repository issues \ghissue{3}
  and \ghissue{4}).  So \colorframed |0.9b| provides
  |\colorframedTitleBarFrame|, which can be used as a replacement to
  |\TitleBarFrame| in a redefined custom `titled-framed' environment.  Check
  the source code for comments.  In particular there is an added macro
\begin{verbatim}
\colorframedTFconlabsep
\end{verbatim}
  which one can define to expand to some dimension which will be the separation
  of the continuation label from the right edge of the frame.  It defaults to |3pt|.

  This variant is still \emph{experimental}.  It is provided as an exception
  to the general rule that this package should take care only of fixing colour
  problems (addition of \emph{new} environments may be considered in future,
  but fixing existing environments will not be done beyond what is described here).

\end{snugshade}
\end{itemize}
\end{footnotesize}


The aim of \colorframed regarding \ctanpackage{framed} existing
code base is strictly limited to fixing the colour leak issues,
there is no general intent to modify the existing environments of
\ctanpackage{framed} for them to acquire additional capabilities and
customizability.

\begin{footnotesize}
  Despite the fact already
  mentioned that anyhow `titled-frame' had been broken some years ago by an
  upstream \LaTeX\ change we did not change it beyond
  fixing this, and the alternative |\colorframedTitleBarFrame| must be
  opted-for explicitly by user in a re-definition of `titled-frame'
  environment, the default remaining with the problematic
  |\TitleBarFrame|. Or the user can do |\let\TitleBarFrame\colorframedTitleBarFrame|.
  \par
\end{footnotesize}
% \color{RawSienna}

Already we have mentioned
\begin{verbatim}
\colorframedTFconlabcolorcommand
\end{verbatim}
whose default definition let it expands to nothing at all (the continuation label
then inherits the |TFFrameColor| colour).

Regarding the `framed' environment, \colorframed adds the
\begin{verbatim}
\colorframedbordercolorcommand
\end{verbatim}
which expands by default to |\normalcolor|.  It influences the colour of the
framing done by environment `framed', the original having no such
customizability.  Do not redefine it to do nothing, it always should set some
colour, else the borders may display colour loss after a page break.

For example
\begin{verbatim}
\setlength{\FrameRule}{5pt}
\setlength{\FrameSep}{9pt}
\renewcommand\colorframedbordercolorcommand{\color{gray!50}}
\end{verbatim}
configures the next usage of `framed'.

\setlength{\FrameRule}{5pt}
\setlength{\FrameSep}{9pt}
\renewcommand\colorframedbordercolorcommand{\color{gray!50}}
\begin{framed}
  This environment allows indeed customization of the border width
  and of the separation with contents, via |\FrameRule| and
  |\FrameSep|.

  The length |\FrameSep| influences also `oframed', `shaded', and `shaded*',
  but `snugshade' and `snugshade*' employ |\fboxsep| rather.

  |\FrameRule| influences both `framed' and `oframed'.
\end{framed}

\section{TODO}

\begin{snugshade}
The author has developed based upon usage of \ctanpackage{pict2e}
breakable boxes with round corners, background colour, optional
shadow (possibly inset), and other goodies and was planning on
incorporating this environment into the package at some time in future.
%
Of course, it will remain limited in comparison to the fully
customizable boxes provided by package \ctanpackage{tcolorbox} (ot
those of \ctanpackage{mdframed}) but our testing showed
significant speed-up in build time, which may matter for long
documents.

% Hopefully this addition will be done when time will permit.

% \thispagestyle{empty}
% \enlargethispage{2\baselineskip}

%\colorlet{shadecolor}{lightgray!50}
% \footnotesize
  But after initial release made it to CTAN on
  2022/09/22 I became aware of \ctanpackage{longfbox} which
  provides already such \ctanpackage{pict2e} breakable boxes with
  rounded corners (even elliptical arcs), and furthermore with a
  CSS-like interface which is exactly what I had done on my side
  too... I need to check more \ctanpackage{longfbox} before a
  decision is made here!  Perhaps it will be better to keep
  \colorframed as it is currently and
  extras such as new \ctanpackage{pict2e}-based boxes with a key-value
  interface to `inline' or `display' boxing macros and environments
  should make it to another package (loading \colorframed of course).

\end{snugshade}
\centerline{\hrulefill documentation ends here\hrulefill}
\end{document}